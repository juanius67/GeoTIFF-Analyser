%NO PREAMBLE, this report is included in a "main.tex" you dont have access to

% %%%%%%%%%%%%%%%%%%%%%%%%%%%%%%%%%%%%%%%%%%%%%%%%%%%%%%%
% % BELOW ARE ALL THE PREAMBLE SETTINGS FOR THE REPORT
% %%%%%%%%%%%%%%%%%%%%%%%%%%%%%%%%%%%%%%%%%%%%%%%%%%%%%%%
% \documentclass{article}

% % --- PACKAGES ---
% \usepackage[utf8]{inputenc}
% \usepackage{graphicx}
% \usepackage{booktabs}
% \usepackage{geometry}
% \usepackage{float}
% \usepackage{fancyhdr}
% \usepackage{amsmath}
% \usepackage{siunitx}

% % --- PAGE LAYOUT ---
% \geometry{margin=2.5cm}

% % --- HEADER & FOOTER ---
% \pagestyle{fancy}
% \fancyhf{}
% \fancyhead[L]{SoilGrids Analysis}
% \fancyfoot[L]{Assignment: Nile Delta vs Desert}
% \fancyfoot[C]{Group 13} 
% \fancyfoot[R]{Page \thepage}
% \renewcommand{\headrulewidth}{0.4pt}
% \renewcommand{\footrulewidth}{0.4pt}

% % --- TITLE ---
% \title{\textbf{The Gift of the Nile: A Quantitative Comparison of Soil Texture between the Delta and the Western Desert}}
% \author{Your Names Here}
% \date{\today}
% %%%%%%%%%%%%%%%%%%%%%%%%%%%%%%%%%%%%%%%%%%%%%%%%%%%%%%%
\section{Assignment 4: Comparison of Soil Texture between Delta and Western Desert}
% %=====================================================
\subsection{1. Introduction}

Egypt has historically been defined by the contrast between the fertile "Black Land" of the Nile floodplain and the arid "Red Land" of the desert. The Egyptian civilization emerged along the Nile river for the obvious reason of the easiness of agriculture with quite special characteristics. The river transports nutrient-rich sediment, mainly silt,  from the Ethiopian highlands, depositing it in on the banks of the Nile, ending up at the Delta. In this report, we use the SoilGrids \cite{SoilGrids} platform to quantitatively analyse this contrast. 

\textbf{Research Question:} \textit{How does soil texture (specifically Clay vs. Sand content) differ quantitatively between the Nile Delta and the surrounding Western Desert?}

\textbf{Hypothesis:} We hypothesize that the Nile Delta will exhibit significantly higher clay concentrations due to alluvial deposition, whereas the Western Desert will be predominantly sand. We aim to quantify this ratio using spatial data analysis.

%=====================================================
\subsection{2. Methods}

\subsection*{Data Acquisition}
We utilized the SoilGrids interface (ISRSC) to download raster data for the region of Egypt. The specific datasets selected were:
\begin{itemize}
    \item \textbf{Variables:} Clay content (\texttt{clay}) and Sand content (\texttt{sand}) in \si{g/kg}.
    \item \textbf{Depth:} Topsoil layer (0--5 \si{cm}).
    \item \textbf{Metric:} Mean prediction (P50).
\end{itemize}

\subsection*{Spatial Analysis}
The downloaded GeoTIFF files were processed using a Python script (utilizing the \texttt{rasterio} and \texttt{numpy} libraries). We defined two specific bounding box regions of interest (ROI) to extract pixel values:
\begin{enumerate}
    \item \textbf{Nile Delta ROI:} Two locations centered at 31.0$^{\circ}$N, 31.0$^{\circ}$E and 30.5$^{\circ}$N, 31.2$^{\circ}$E. These locations were chosen to represent the fertile alluvial soil deposited by the Nile, which is the center of Egyptian agriculture.
    \item \textbf{Western Desert ROI:} Two locations centered at 30.8$^{\circ}$N, 30.2$^{\circ}$E and 29.8$^{\circ}$N, 31.6$^{\circ}$E. These locations were selected as a control group to represent the arid, sandy bedrock and dunes typical of the Sahara, providing a sharp contrast to the Delta.
\end{enumerate}
We calculated the mean mass fraction for sand and clay in both regions and visualized the comparison using histograms.

%=====================================================
\subsection{3. Results}

\subsection*{Texture Composition}

The analysis revealed a stark contrast in soil composition. 

\begin{figure}[H]
    \centering
    \includegraphics[width=0.8\textwidth]{soil_composition_comparison.png}
    \caption{Comparison of Mean Sand vs. Clay content (\%) in the Nile Delta compared to the Western Desert.}
    \label{fig:barchart}
\end{figure}

As shown in Figure \ref{fig:barchart}, the Western Desert is composed of significantly more sand (Mean $\approx$ 37.61\%). In contrast, the Nile Delta shows a much higher proportion of clay (Mean $\approx$ 27.62\%) compared to the desert (Mean $\approx$ 30.10\% clay, though this value is unexpectedly high, likely due to the specific sampling locations near transition zones or oases).

\begin{figure}[H]
    \centering
    \includegraphics[width=0.9\textwidth]{combined_soil_maps.png}
    \caption{Spatial distribution of Clay and Sand content across the four sampled locations. Note the higher clay density (darker orange) in the Nile Delta samples compared to the Western Desert samples.}
    \label{fig:maps}
\end{figure}

Table \ref{tab:results} summarizes the exact values obtained from the analysis:

\begin{table}[H]
    \centering
    \begin{tabular}{lcccc}
        \toprule
        \textbf{Group} & \textbf{Location} & \textbf{Coords} & \textbf{Clay (\%)} & \textbf{Sand (\%)} \\
        \midrule
        Nile Delta & Delta 1 & 31.0N, 31.0E & 31.87 & 35.29 \\
        Nile Delta & Delta 2 & 30.5N, 31.2E & 23.37 & 30.96 \\
        \midrule
        Western Desert & Desert 1 & 30.8N, 30.2E & 25.61 & 41.85 \\
        Western Desert & Desert 2 & 29.8N, 31.6E & 34.59 & 33.37 \\
        \bottomrule
    \end{tabular}
    \caption{Summary of soil texture analysis results.}
    \label{tab:results}
\end{table}

%=====================================================
\subsection{Discussion}

The results quantitatively confirm the geological distinction between the two regions, although with some variability. The Nile Delta samples consistently show significant clay content, aligning with the alluvial deposition hypothesis. The Western Desert samples generally show higher sand content, particularly in location 1 (41.85\%).

However, Western Desert 2 showed surprisingly high clay content (34.59\%), which may indicate a sampling location near a depression, oasis, or paleolake deposit rather than active dune fields. This highlights the heterogeneity of the desert landscape.

This difference determines the agricultural viability of the region: clay soils retain water and nutrients, enabling the intensive farming seen in the Delta, while the sandy desert soils drain rapidly and lack nutrient cohesion.

%=====================================================
\subsection{Conclusion}

Using SoilGrids data, we successfully quantified the "Gift of the Nile." The analysis showed that the Delta generally contains distinct soil signatures compared to the surrounding desert. This exercise highlights how remote sensing and global soil datasets can be used to verify geomorphological features without field sampling.

\newpage
\section*{Annex: Location Visualizations}

Below are the satellite views of the specific locations sampled in this analysis.

\begin{figure}[H]
    \centering
    \begin{minipage}{0.45\textwidth}
        \centering
        \includegraphics[width=\linewidth]{LocationNile1.jpg}
        \caption{Nile Delta Location 1}
    \end{minipage}
    \hfill
    \begin{minipage}{0.45\textwidth}
        \centering
        \includegraphics[width=\linewidth]{LocationNile2.jpg}
        \caption{Nile Delta Location 2}
    \end{minipage}
\end{figure}

\begin{figure}[H]
    \centering
    \begin{minipage}{0.45\textwidth}
        \centering
        \includegraphics[width=\linewidth]{LocationDesert1.jpg}
        \caption{Western Desert Location 1}
    \end{minipage}
    \hfill
    \begin{minipage}{0.45\textwidth}
        \centering
        \includegraphics[width=\linewidth]{LocationDesert2.jpg}
        \caption{Western Desert Location 2}
    \end{minipage}
\end{figure}

% \end{document}